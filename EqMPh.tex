\documentclass[11pt,a4paper]{article}
\usepackage[utf8x]{inputenc}
\usepackage{mathtext}
\usepackage[T1]{fontenc}
\usepackage[russian]{babel}
\usepackage{ucs}
\usepackage{amsmath}
\usepackage{amsfonts}
\usepackage{amssymb}
\usepackage{graphicx}
\usepackage{fullpage}
\usepackage[left=0.1cm,right=0.1cm,top=0.1cm,bottom=0.1cm,bindingoffset=0cm]{geometry}
\usepackage{amsthm}


\begin{document}
    \LARGE Билет 10. Принцип максимума для уравнения теплопроводности. Теоремы единственности и устойчивости для смешанной задачи для уравнения теплопроводности. Общее решение уравнения Лапласа в полярных координатах.
    \section{Принцип максимума для уравнения теплопроводности. Теоремы единственности и устойчивости для смешанной задачи для уравнения теплопроводности.}
    \subsection{Принцип максимума для уравнения теплопроводности.}
    
    Рассмотрим множество $Q_T = \{(x, t) : (0; l) × (0; T]\}$. Обозначим $Г = Q_T$ $\overline{Q_T}$.
    \par
    \textbf{Теорема.} \textit{(принцип максимума)} Пусть $u(x,t) \in C[\overline{Q_T}], u_t, u_{xx} \in C[Q_t]$ и $u_t = a^2u_{xx}$. Тогда
    $$
    \underset{\overline{Q_T}}{max} \; u(x,t) = \underset{Г}{max} \; u(x,t)
    $$
    $$
    \underset{\overline{Q_T}}{min} \; u(x,t) = \underset{Г}{min} \; u(x,t)
    $$
    
    \textit{Доказательство.} $\triangleleft$ Сначала докажем утверждение для $max$. Предположим противное: пусть $\underset{Г}{max} \; u(x,t) = M$ и $\exists$ точка $(x_0,t_0) \in Q_T$ такая, что $u(x_0,t_0)= M + \varepsilon, \; \varepsilon > 0$.
    \par
    Тогда введем $v(x,t)$:
    \begin{equation}
        \label{eq1}
        v(x,t) = u(x,t) - \frac{\varepsilon}{2T}(t - t_0)
    \end{equation}
    
    Очевидно, что $v(x_0,t_0) = u(x_0,t_0) = M + \varepsilon$. Так как $|\frac{\varepsilon}{2T}(t - t_0)| \leq \frac{\varepsilon}{2}$ при $t \in [0,T]$, то:
    $$
    \underset{Г}{max} \; v(x,t) = \underset{Г}{max} \{u(x,t) - \frac{\varepsilon}{2T}(t - t_0)\} \leqslant M + \frac{\varepsilon}{2}
    $$
    
    Отсюда следует, что $\exists$ точка $(x_1,t_1) \in Q_T$, в которой $v(x,t)$ достигает максимума. Тогда по необходимому условию максимума дважды дифференцируемой функции получаем:
    \begin{equation}
        \label{eq2}
        \begin{cases}
        v_t(x_1,t_1) \geqslant 0 \\
        v_{xx}(x_1,t_1) \leqslant 0
        \end{cases}
    \end{equation}
    
    Продифференцируем (\ref{eq1}) отдельно один раз по $t$ и отдельно два раза по $x$. Получим:
    \begin{equation}
        \label{eq3}
        v_t(x,t) = u_t(x,t) - \frac{\varepsilon}{2T}
    \end{equation}
    $$
    v_{xx}(x,t) = u_{xx}(x,t)
    $$
    
    Из полученных равенств и системы (\ref{eq2}) следует, что:
    $$
    u_t(x_1,t_1) = v_T(x_1,t_1) + \frac{\varepsilon}{2T} > 0 \geqslant a^2v_{xx}(x_1,t_1) = a^2u_{xx}(x_1,t_1)
    $$
    
    \textit{(В (\ref{eq3}) перенесли $\frac{\varepsilon}{2T} > 0$. Из (\ref{eq2}) следует, что $v_t(x_1,t_1) \geq v_{xx}(x_1,t_1)$. Во все выражения подставили $(x_1,t_1)$. Домножили на $a^2 > 0$ производные по $x$, что не влияет на знак выражения.)}
    
    Получим $u_t(x_1,t_1) > a^2u_{xx}(x_1,t_1)$, что противоречит уравнению теплопроводности. Первое утверждение доказано.
    \par
    Второе утверждение доказывается аналогично заменой $\omega(x,t) = -u(x,t)$ и рассмотрением первого утверждения для $\omega(x,t)$
    \par
    Теорема доказана. $\triangleright$
    
    \subsection{Теорема единственности решения смешанной задачи для уравнения теплопроводности.}
    Смешанная задача для уравнения теплопроводности:
    \begin{equation}
      \label{eq4}
      \begin{cases}
          u_t(x,t) = a^2u_{xx}(x,t), \; 0 \leqslant x \leqslant l, 0 < t \leqslant T \\
          u(0, t) = \mu_1(t) \\
          u(l, t) = \mu_2(t) \\
          u(x, 0) = \phi(x)
      \end{cases}
    \end{equation}
   
    $\forall T > 0$. В общем случае $T = +\infty $
    
    \textbf{Теорема.} \textit{(единственности)} Пусть $u_1(x,t),u_2(x,t)$ являются решениями одной и той же задачи (\ref{eq4}) и $u_i(x,t) \in C[\overline{Q_T}]$ и $(u_i)_{t}^{'}(x,t),(u_i)_{xx}^{''}(x,t) \in C[Q_T], \; \forall T > 0, i = 1,2$. Тогда $u_1(x,t) \equiv u_2(x,t)$.
    \par
    \textit{Доказательство.} $\triangleleft$ Введем функцию $v(x,t) = u_1(x,t) - u_2(x,t)$ такую, что $v(x,t) \in C[\overline{Q_T}], v_t, v_{xx} \in C[Q_t]$. Она является решением краевой задачи:
    $$
    \begin{cases}
        v_t(x,t) = a^2u_{xx}(x,t), \; 0 \leqslant x \leqslant l, 0 < t \leqslant T \\
        v(0, t) = 0 \\
        v(l, t) = 0 \\
        v(x, 0) = 0
    \end{cases}
    $$
    
    Для $v(x,t)$ выполнены все условия принципа максимума. Тогда:
    $$
    \begin{cases}
        \underset{\overline{Q_T}}{max} \; v(x,t) = \underset{Г}{max} \; v(x,t) = 0 \\
        \underset{\overline{Q_T}}{min} \; v(x,t) = \underset{Г}{min} \; v(x,t) = 0
    \end{cases}
    \Rightarrow v(x,t) \equiv 0 \Rightarrow u_1(x,t) \equiv u_2(x,t)
    $$
    
    Теорема доказана. $\triangleright$
    
    \subsection{Теорема устойчивости решения смешанной задачи для уравнения теплопроводности}
    \textbf{Лемма.} Если $u_1(x,t),u_2(x,t)$ такие, что $u_i(x,t) \in C[\overline{Q_T}]$ и $(u_i)_{t}^{'}(x,t),(u_i)_{xx}^{''}(x,t) \in C[Q_T], \; \forall T > 0, i = 1,2$ и являются решениями разных задач (\ref{eq4}), причем все граничные условия задачи для $u_1(x,t)$ больше или равны граничным условиям задачи для $u_2(x,t)$, то $u_1(x,t) \geqslant u_2(x,t)$ в $\overline{Q_T}$.
    
    \textit{Доказательство.} $\triangleleft$ Введем функцию $v(x,t) = u_1(x,t) - u_2(x,t)$ такую, что $v(x,t) \in C[\overline{Q_T}], v_t, v_{xx} \in C[Q_t]$. Она является решением краевой задачи:
    $$
    \begin{cases}
        v_t(x,t) = a^2u_{xx}(x,t), \; 0 \leqslant x \leqslant l, 0 < t \leqslant T \\
        v(0, t) \geqslant 0 \\
        v(l, t) \geqslant 0 \\
        v(x, 0) \geqslant 0
    \end{cases}
    $$
    
    Для $v(x,t)$ выполнены все условия принципа максимума. Тогда:
    
    $$
    \underset{\overline{Q_T}}{min} \; v(x,t) = \underset{Г}{min} \; v(x,t) \geqslant 0 \Rightarrow u_1(x,t) \geqslant u_2(x,t), \; \forall (x,y) \in \overline{Q_T}
    $$
    
    Лемма доказана. $\triangleright$
    \par
    \textbf{Теорема.} \textit{(устойчивости)} Если $u_1(x,t),u_2(x,t)$ такие, что $u_i(x,t) \in C[\overline{Q_T}]$ и $(u_i)_{t}^{'}(x,t),(u_i)_{xx}^{''}(x,t) \in C[Q_T], \; \forall T > 0, i = 1,2$ и являются решениями разных задач (\ref{eq4}), причем все граничные условия двух задач различаются по модулю $< \varepsilon$ ($|u_1(0,t) - u_2(0,t)| \leqslant \varepsilon$ и так для каждого граничного условия), то 
    $$
    \underset{\overline{Q_T}}{max} \; |u_1(x,t) - u_2(x,t)| \leqslant \varepsilon
    $$
    
    \textit{Доказательство.} $\triangleleft$ Введем функцию $v(x,t) = u_1(x,t) - u_2(x,t)$ такую, что $v(x,t) \in C[\overline{Q_T}], v_t, v_{xx} \in C[Q_t]$. Тогда $v_t(x,t) = a^2v_{xx}(x,t)$ и для нее выполняются все условия принципа максимума. Получим:
    $$
    \underset{\overline{Г}}{max} \; |v(x,t)| \leqslant \varepsilon
    $$
    
    То есть $-\varepsilon \geqslant v(x,t) \leqslant \varepsilon$ на Г (границе $Q_T$). Применив лемму к парам функций ($-\varepsilon, v(x,t)$) и ($v(x,t), \varepsilon$), получим:
    $$
    -\varepsilon \geqslant u_1(x,t) - u_2(x,t) \leqslant \varepsilon \; \forall (x,y) \in \overline{Q_T}
    $$
    Теорема доказана. $\triangleright$
    
    Полученное утверждение означает, что из близости исходных данных следует близость полученных
решений.
    
    \section{Общее решение уравнения Лапласа в полярных координатах.}
    Уравнение Лапласа: $\Delta u = 0 \Leftrightarrow \frac{1}{r} \frac{\partial}{\partial r}(r \frac{\partial u}{\partial r}) + \frac{1}{r^2} \frac{\partial^2 u}{\partial \phi^2} = 0$ \textit{(оператор Лапласа для полярных координат)}
    Функция $u(r,\phi)$ ищется в виде: $\sum\limits_{n=0}^{\infty} R_n(r)\Phi_n(\phi)$
    Подставим $R(r)$ и $\Phi(\phi)$ в оператор Лапсласа для полярных координат:
    $$
    \frac{1}{r} \frac{d}{dr}(r \frac{dR}{dr})\Phi + \frac{1}{r^2} R \frac{d^2 \Phi}{d \phi^2} = 0 \; |:\Phi \; \Rightarrow \;
    \frac{r(rR')'}{R} + \frac{\Phi''}{\Phi} = 0
    $$
    Решаем два уравнения для $R_n(r)$ и $\Phi_n(\phi)$. Не забыть про $n = 0$. В итоге получим общее решение для $u(r, \phi)$:
    $$
    u(r, \phi) = a_0 + b_0 \, ln\:r + \sum\limits_{n=1}^{\infty}r^n(a_n \, cos \: n \phi + b_n \, sin \: n \phi) + \sum\limits_{n=1}^{\infty}\frac{1}{r^n}(c_n \, cos \: n \phi + d_n \, sin \: n \phi)
    $$
    
\end{document}