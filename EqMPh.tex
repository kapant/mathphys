\documentclass[a6paper]{article}
\usepackage[margin=3mm]{geometry}
\usepackage[utf8x]{inputenc}
\usepackage{mathtext}
\usepackage[T1]{fontenc}
\usepackage[russian]{babel}
\usepackage{ucs}
\usepackage{amsmath}
\usepackage{amsfonts}
\usepackage{amssymb}
\usepackage{graphicx}
\usepackage{indentfirst}


\title{Билет 10}
\date{}

\begin{document}
    \maketitle
    \section{Принцип максимума для уравнения теплопроводности. Теоремы единственности и устойчивости для смешанной задачи для уравнения теплопроводности.}
    \subsection{Принцип максимума для уравнения теплопроводности.}
    
    Рассмотрим множество $Q_T = \{(x, t) : (0; l) × (0; T]\}$. Обозначим $Г = Q_T$ $\overline{Q_T}$.
    \par
    \textbf{Теорема.} \textit{(принцип максимума)} Пусть $u(x,t) \in C[\overline{Q_T}], u_t, u_{xx} \in C[Q_t]$ и $u_t = a^2u_{xx}$. Тогда
    $$
    \underset{\overline{Q_T}}{max} \; u(x,t) = \underset{Г}{max} \; u(x,t)
    $$
    $$
    \underset{\overline{Q_T}}{min} \; u(x,t) = \underset{Г}{min} \; u(x,t)
    $$
    
    \textit{Доказательство.} Сначала докажем утверждение для $max$. Предположим противное: пусть $\underset{Г}{max} \; u(x,t) = M$ и $\exists$ точка $(x_0,t_0) \in Q_T$ такая, что $u(x_0,t_0)= M + \varepsilon, \; \varepsilon > 0$.
    \par
    Тогда введем $v(x,t)$:
    \begin{equation}
        \label{eq1}
        v(x,t) = u(x,t) - \frac{\varepsilon}{2T}(t - t_0)
    \end{equation}
    
    Очевидно, что $v(x_0,t_0) = u(x_0,t_0) = M + \varepsilon$. Так как $|\frac{\varepsilon}{2T}(t - t_0)| \leq \frac{\varepsilon}{2}$ при $t \in [0,T]$, то:
    $$
    \underset{Г}{max} \; v(x,t) = \underset{Г}{max} \{u(x,t) - \frac{\varepsilon}{2T}(t - t_0)\} \leq M + \frac{\varepsilon}{2}
    $$
    
    Отсюда следует, что $\exists$ точка $(x_1,t_1) \in Q_T$, в которой $v(x,t)$ достигает максимума. Тогда по необходимому условию максимума дважды дифференцируемой функции получаем:
    \begin{equation}
        \label{eq2}
        \begin{cases}
        v_t(x_1,t_1) \geq 0 \\
        v_{xx}(x_1,t_1) \leq 0
        \end{cases}
    \end{equation}
    
    Продифференцируем (\ref{eq1}) отдельно один раз по $t$ и отдельно два раза по $x$. Получим:
    \begin{equation}
        \label{eq3}
        v_t(x,t) = u_t(x,t) - \frac{\varepsilon}{2T}
    \end{equation}
    $$
    v_{xx}(x,t) = u_{xx}(x,t)
    $$
    
    Из полученных равенств и системы (\ref{eq2}) следует, что:
    $$
    u_t(x_1,t_1) = v_T(x_1,t_1) + \frac{\varepsilon}{2T} > 0 \geq a^2v_{xx}(x_1,t_1) = a^2u_{xx}(x_1,t_1)
    $$
    
    \textit{(В (\ref{eq3}) перенесли $\frac{\varepsilon}{2T} > 0$. Из (\ref{eq2}) следует, что $v_t(x_1,t_1) \geq v_{xx}(x_1,t_1)$. Во все выражения подставили $(x_1,t_1)$. Домножили на $a^2 > 0$ производные по $x$, что не влияет на знак выражения.)}
    
    Получим $u_t(x_1,t_1) > a^2u_{xx}(x_1,t_1)$, что противоречит уравнению теплопроводности. Первое утверждение доказано.
    \par
    Второе утверждение доказывается аналогично заменой $\omega(x,t) = -u(x,t)$ и рассмотрением первого утверждения для $\omega(x,t)$
    \par
    Теорема доказана.
    
    
    
    \section{Общее решение уравнения Лапласа в полярных координатах.}
    
\end{document}